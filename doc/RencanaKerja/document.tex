\documentclass[a4paper,twoside]{article}
\usepackage[T1]{fontenc}
\usepackage[bahasa]{babel}
\usepackage{graphicx}
\usepackage{graphics}
\usepackage{float}

\usepackage{hyperref}
\usepackage[dvipsnames]{xcolor}
\definecolor{myred}{rgb}{0.890,0.102,0.110} %227,26.28
\definecolor{myblue}{rgb}{0.122,0.471,0.706} %31,120,180
\definecolor{mygreen}{rgb}{0.20,0.627,0.173} % 51,160,44
\definecolor{myorange}{rgb}{0.929, 0.459, 0.219} %237, 117, 56
\hypersetup{colorlinks=true,urlcolor=myblue,linkcolor=mygreen}

\usepackage[cm]{fullpage}
\pagestyle{myheadings}
\usepackage{etoolbox}
\usepackage{setspace} 
\usepackage{lipsum} 
\setlength{\headsep}{30pt}
\usepackage[inner=2cm,outer=2.5cm,top=2.5cm,bottom=2cm]{geometry} %margin
% \pagestyle{empty}

\makeatletter
\renewcommand{\@maketitle} {\begin{center} {\LARGE \textbf{ \textsc{\@title}} \par} \bigskip {\large \textbf{\textsc{\@author}} }\end{center} }
\renewcommand{\thispagestyle}[1]{}
\markright{\textbf{\textsc{AIF184001 \textemdash Rencana Kerja Skripsi \textemdash Sem. Genap 2021/2022}}}

\newcommand{\oj}{\textit{online judge}}

\newcommand{\HRule}{\rule{\linewidth}{0.4mm}}
\renewcommand{\baselinestretch}{1}
\setlength{\parindent}{0 pt}
\setlength{\parskip}{6 pt}

\onehalfspacing
 
\begin{document}

\title{\@judultopik}
\author{\nama \textendash \@npm} 

%tulis nama dan NPM anda di sini:
\newcommand{\nama}{Edwin Pranajaya}
\newcommand{\@npm}{2017730027}
\newcommand{\@judultopik}{Dukungan Bahasa JavaScript pada SharIF Judge} % Judul/topik anda
\newcommand{\jumpemb}{1} % Jumlah pembimbing, 1 atau 2
\newcommand{\tanggal}{22/02/2022}

% Dokumen hasil template ini harus dicetak bolak-balik !!!!

\maketitle

\pagenumbering{arabic}

\section{Deskripsi}

\textit{Online judge} merupakan sebuah sistem yang dirancang untuk melakukan evaluasi dari sumber kode algoritme yang dikirimkan oleh pengguna. Konsep dari \oj adalah dengan mengasumsi pengguna mengirimkan solusi berupa kode program, atau bahkan pengguna mengirimkan file yang dapat dijalankan dimana tahap selanjutnya kode ato file tersebut akan dievaluasi yang sering kali menggunakan infrastruktur berbasis \textit{cloud}. Dalam perancangan \oj harus dipastikan bahwa \oj harus dapat menanggulangi berbagai macam serangan seperti memaksakan waktu kompilasi yang tinggi, atau mengakses sumber daya yang dibatasi selama proses evaluasi berlangsung.

Sharif Judge merupakan salah satu \oj yang gratis untuk bahasa pemrograman C, C++, Java dan Python \footnote{https://github.com/mjnaderi/Sharif-Judge}. Perangkat lunak ini diciptakan oleh Mohammad Javad Naderi pada tahun 2014 dan bersifat open source. Antarmuka Sharif Judge ditulis menggunakan bahasa pemrograman PHP (framework CodeIgniter) dan backend menggunakan BASH.

Sharif Judge biasa digunakan untuk mempermudah evaluasi kode program. dan salah satu universitas yang menggunakannya adalah Universitas Katolik Parahyangan (UNPAR) pada jurusan Informatika (IF). Namun Sharif Judge telah dimodifikasi dan berubah namanya menjadi SharIF-Judge. SharIF-Judge ini digunakan pada beberapa kuliah di IF, seperti Dasar Pemrograman dan Algoritma Struktur Data. Tujuan SharIF-Judge digunakan pada perkuliahan, adalah agar dapat mempermudah kegiatan belajar mengajar berlangsung. Mahasiswa dapat dengan mudah melakukan pengiriman kode program atau file ke SharIF-Judge dan dapat langsung melihat nilainya. Pengajar yang bertanggung jawab atas kegiatan belajar mengajar tersebut tidak akan kesusahan dalam mengumpulkan hasil pekerjaan para pelajar. 

Alur dari penggunaan SharIF-Judge ini diawali dengan pengajar yang membuat soal terlebih dahulu. Setelah soal disiapkan, pengajar dapat membuat \textit{assignment} dengan click tombol \textit{add assignment}. Didalamnya, pengajar diwajibkan untuk memasukan nama \textit{assignment}, waktu dimulainya pengerjaan, waktu selesai pengerjaan, waktu tambahan pengerjaan, daftar peserta, deskripsi soal, dan kunci jawaban dari soal yang sudah dibuat oleh pengajar. Meskipun SharIF-Judge dapat membaca berbagai bahasa pemrograman, JavaScript bukan salah satu yang dapat dibaca oleh SharIF-Judge.

JavaScript adalah bahasa skrip sisi klien yang berjalan sepenuhnya di dalam browser web. Sebagai contoh dapat dilihat \textit{menu drop-down} yang mencolok, memindahkan teks, dan mengubah konten yang sekarang tersebar luas di situs web. Semua interaksi tersebut diaktifkan melalui JavaScript. berdasarkan survey dari \url{https://insights.stackoverflow.com/survey/2021}, selama 9 tahun berturut-turut, JavaScript merupakan bahasa pemrograman yang paling sering digunakan. 64,96\% orang di dunia menggunakannya.

Berdasarkan data diatas, ditunjukan bahwa JavaScript termasuk dalam bahasa pemrograman yang sangat populer dan banyak digunakan. Sangat disayangkan jika SharIF Judge tidak dapat membaca bahasa tersebut. Akan kesulitan bagi para pengajar karena harus mengikuti perkembangan zaman, namun pemeriksaannya tidak mudah. Kostumisasi dari SharIF Judge akan dilakukan agar \textit{online judge} tersebut dapat membaca JavaScript. Untuk mempermudah kostumasi, pengerjaan akan dilakukan dengan menggunakan OS Linux. Namun dikarenakan OS yang digunakan adalah Windows, akan digunakan \textit{virtual machine} agar dapat mengerjakan dengan OS Linux pada Windows. 
%Sehingga, ketika pengajar membuatkan soal tentang JavaScript , pengajar dapat membuatkan soal dan tes kasus terhadap JavaScript. Para pelajar pun dapat dengan mudah mengetahui apakah kode yang mereka buat tentang JavaScript sudah sesuai atau belum. Dengan JavaScript dapat dibaca oleh SharIF Judge, akan dengan mudah dan cepat evaluasi terhadap pekerjaan para pelajar.


\section{Rumusan Masalah}
\label{sec:rumusan}
Berdasarkan latar belakang tersebut, maka terbentuklah rumusan masalah penelitian sebagai berikut: 

     Bagaimana cara agar SharIF Judge dapat memahami bahasa pemrograman JavaScript untuk dievaluasi.


\section{Tujuan}
\label{sec:tujuan}
Berdasarkan rumusan masalah, maka tujuan penelitian ini adalah sebagai berikut:

    Memodifikasi SharIF Judge agar dapat menerima soal JavaScript dan dapat mengevaluasi JavaScript.


\section{Deskripsi Perangkat Lunak}
Perangkat lunak akhir yang akan dibuat memiliki fitur minimal sebagai berikut:
\begin{itemize}
	\item Perangkat lunak dapat menjalankan fitur yang sebelumnya dimiliki.
	\item Perangkat lunak dapat melakukan evaluasi terhadap JavaScript.
		
\end{itemize}

\section{Detail Pengerjaan Skripsi}
Bagian-bagian pekerjaan skripsi ini adalah sebagai berikut :
	\begin{enumerate}
		\item Melakukan pembelajaran terhadap kode SharIF Judge
		\item Melakukan studi literatur terhadap JavaScript
		\item Melakukan studi literatur mengenai ECMAscript
		\item Merancang fitur yang akan diimplementasikan
		\item Implementasi fitur terhadap perangkat lunak
		\item Menguji perangkat lunak ke mata kuliah selama 1 semester
		\item Membuat dokumentasi perangkat lunak
		\item Menulis dokumen skripsi
	\end{enumerate}

\section{Rencana Kerja}
Rincian capaian yang direncanakan di Skripsi 1 adalah sebagai berikut:
\begin{enumerate}
\item Mempelajari kode program SharIF Judge
\item Mempelajari bagaimana program tersebut dapat membaca bahasa pemrograman yang sudah ada.
\item Mencoba mengimplementasikan dari kode program yang sudah ada agar dapat membaca javascript 
\item Menuliskan dokumen
\end{enumerate}

Sedangkan yang akan diselesaikan di Skripsi 2 adalah sebagai berikut:
\begin{enumerate}
\item Melakukan testing SharIF-Judge yang telah di kostumisasi kepada mata kuliah yang ada pada jurusan Infromatika
\item Melakukan \textit{bug fixing} jika ada \textit{bug}
\item Menuliskan dokumen skripsi
\end{enumerate}

\vspace{1cm}
\centering Bandung, \tanggal\\
\vspace{2cm} \nama \\ 
\vspace{1cm}

Menyetujui, \\
\ifdefstring{\jumpemb}{2}{
\vspace{1.5cm}
\begin{centering} Menyetujui,\\ \end{centering} \vspace{0.75cm}
\begin{minipage}[b]{0.45\linewidth}
% \centering Bandung, \makebox[0.5cm]{\hrulefill}/\makebox[0.5cm]{\hrulefill}/2013 \\
\vspace{2cm} Nama: \makebox[3cm]{\hrulefill}\\ Pembimbing Utama
\end{minipage} \hspace{0.5cm}
\begin{minipage}[b]{0.45\linewidth}
% \centering Bandung, \makebox[0.5cm]{\hrulefill}/\makebox[0.5cm]{\hrulefill}/2013\\
\vspace{2cm} Nama: \makebox[3cm]{\hrulefill}\\ Pembimbing Pendamping
\end{minipage}
\vspace{0.5cm}
}{
% \centering Bandung, \makebox[0.5cm]{\hrulefill}/\makebox[0.5cm]{\hrulefill}/2013\\
\vspace{2cm} Nama: \makebox[3cm]{\hrulefill}\\ Pembimbing Tunggal
}
\end{document}
